%=== FRONT PART ===
%=== ABSTRCT ===

%\begin{center}
\chapter*{Explanations}
%\rhead{Abstract}
%\end{center}
\addcontentsline{toc}{chapter}{Abstract}

\section*{Assumptions}

\begin{itemize}
	\item For the \textbf{\textit{User}} entity, \texttt{date of birth}, \texttt{name}, and \texttt{gender} are optional. If we don’t have information about it, ``\texttt{None}'' will be filled in.
	\item For the \textit{\textbf{User Preference}}, ``\texttt{learned information}'' uses natural language to store the user’s preference, generated by ChatGPT.
	\item Assume that \textit{\textbf{Cafeterias}} are a type of \textit{\textbf{Restaurant}} and, hence can be grouped under the same entity set \textit{\textbf{Restaurant \& Cafeterias}}.
	\item Assume that a \textit{\textbf{User}} can sign up for and be part of multiple \textit{\textbf{Day Packages}} at the same time.
	\item Assume that \textit{\textbf{Cash Vouchers}} can only be used in \textit{\textbf{Restaurants \& Cafeterias}}, and \textit{\textbf{Purchase Vouchers}} can only be used in \textit{\textbf{Retail Shops}}.
	\item Assume that \textit{\textbf{Discount Vouchers}} mentioned in Appendix A of the Lab Manual, bullet point 6 is not another type of voucher, but refer to either of the 2 existing types of voucher, \textit{\textbf{Purchase Voucher}} or \textit{\textbf{Cash Voucher}}.
\end{itemize}

\section*{Features}

\begin{itemize}
	\item The system can recommend suitable restaurants according to the user's visiting history.
	\item For the user-history-related queries (like the earnings of the restaurant), we only maintain the user history instead of maintaining all the answers for queries, which follows the principle ``\textit{\textbf{Avoid Redundancy}}''.
	\item Instead of creating different types of vouchers, we use a general entity Vouchers with an attribute type, following the design principle ``\textit{\textbf{Keep It Simple}}''.
\end{itemize}